\thispagestyle{empty} 
\begin{abstract}
In this thesis, multiple imputation, survival analysis, and propensity score analysis are combined in order to answer questions about cancer data with moderate missingness. While each of these fields have been studied individually, there has been little work and analysis on using the three in trio. Starting with an incomplete dataset, the goal is to impute the missing data, run survival analysis on each of the imputed datasets, and then do propensity score analysis to observe causal effects.  Along the way, many theoretical and analytical decisions are made. I explain why each decision is made, and offer ample evidence for the other choices such that the interested reader may implement the methods if they so choose. I apply the methodology to a cancer survival dataset in a case study, but the methods used are general, and could be adapted for any type of data.
\end{abstract}
