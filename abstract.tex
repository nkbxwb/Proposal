\thispagestyle{empty} 
\begin{abstract}
In this thesis, we will use the theory of multiple imputation, survival analysis, and propensity score analysis in order to answer questions about cancer survival data. While each of these fields have been studied individually, there has been little work and analysis on using the three in trio. Starting out with an incomplete dataset, we aim to impute reasonable imputations, run survival analysis on each of the imputed datasets to get survival estimates, and then do propensity score analysis to observe causal effects.  Along the way, many theoretical and analytical decisions are mode. I explain why each decision is made, but offer ample evidence for the other choices such that the interested reader may implement the methods if they so choose. I apply the methodology to a cancer survival dataset in a case study, but the methods used are general, and could be used for any type of data.
\end{abstract}

