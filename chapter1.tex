\chapter{Introduction}
\section{Motivation}
\label{sec:Motivation}
I want to develop methodology that can be used both by applied researchers and clinicians. 
I want it to be easy enough to describe to someone with a limited statistical background, 
but meaningful and valid so that the results obtained can be used in publication. 
The desire to have it this way stems from working on a project with both statisticians and clinicians.

\label{ch:Intro}
Missing data is a major problem in both applied and theoretical statistics, however, 
it has not received attention proportional to its need. Survival analysis is a well studied field, 
but is relatively complete, so not much new research comes out of this field. 
Propensity score analysis will help us determine causal relationships when we don�t have a completely
randomized experiment. As one could imagine, all three of these fields are important to the applied statistician,
as they will come across at least one at some point in their career.  
I will explain each of these three disciplines in detail before we dive into combining them.

\section{Imputation}
\label{sec:Imputation}
Imputation (specifically multiple imputation) is a way to �fill in missing data�, and it forms the base of our plan, all of the other analyses follow from it. Thus, we need a good understanding of it before we may proceed. Multiple imputation itself though is a recent development, so we need to understand why it is used.
At first, statisticians payed no attention to missing data, and happily discarded records for their analysis that were incomplete. This was known as complete case analysis. There are many problems with this paradigm. To begin with, you will lose a lot of statistical power when doing this, because you are literally throwing away records and thus decreasing your sample size.  In addition, this can be costly to the researcher. If it costs a set amount to collect a single record, and you don�t use this record, you are literally wasting money. As well, in some rare cases, incomplete data might be the only type we can get. Lastly, and most importantly, we will be biasing our estimates if we discard them. For example, if we have a random sample of people and are testing a drug, and want to run a regression on some collected covariates. Men are known to not want to give all of their information, so they leave them blank. In the analysis, we will need to discard the male samples because they are incomplete, leaving us only with women. Thus, we don�t have a random sample anymore, and will get biased results.
The next wave of statisticians wanted to improve upon this, so they developed what we now call today (single) imputation. In single method (such as regression, trees) is used one time to impute the missing value. While this is a little better than  complete case analysis, it still has many drawbacks. Asserting that a single value is the true value is just foolish. There is always some amount of error involved, and we can in no way be 100% confident that our imputed value is correct. Furthermore, if I impute one value and you impute another, we may get totally different results from analysis on the data. This is obviously not desirable. 
