\chapter{Conclusion}
This paper details how to use multiply imputed data to answer survival and causal analysis questions. The motivation for the methods used is cancer data, although sufficient detail is given so that the methods can be applied towards other areas. The first section gives background information about multiple imputation, survival analysis, and causal analysis. In the second section, methods and theory are presented to prepare for a cancer data analysis. In the third section, we test the methods out on a large cancer dataset, trying to draw meaningful inference from a dataset with substantial missingness. It is determined in many different ways that for cancer patients, other chemotherapeutic drugs are better in terms of  survival than Capecitabine, but that taking any chemotherapeutic drugs is better than not taking any at all. For HER2+ patients, it is shown that Lapatinib and Trastuzumab are about the same in terms of survival, but both are far superior than no HER2-directed treatment.
