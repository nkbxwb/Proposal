\chapter{Application}
\section{Data Explanation}
\label{sec:data}
\begin{comment}
We have now laid down the theory of what we want to do describing all of the choices we may need to make, so now we will put it in action. The data that we will apply it to is a dataset from M.D Anderson Cancer Center. The data is a collection of about 1500 patients at MD Anderson who have breast cancer that has metastasized to the brain, about 100 clinically relevant covariates, along with their survival status and time.
\end{comment}
Now that the theory is in place, we can apply it to real data. The dataset that I chose to analyze is a dataset from MD Anderson Cancer Center, with permission from Dr. Bugano, Dr. Ibrahim, and Dr. Hess. The IRB protocol is RCR03-0931. This dataset has historical records of 1514 MD Anderson patients who have had breast cancer that has metastasized to the brain from October 2009 to December 2012. The data is not from an RCT, rather it is from a retrospective observational study of patient records, where oncologists chose what treatments to give. The word metastases is often shortened in speech and in paper to the word ``mets'' because it is easier to say.

The dataset consists of 111 covariates, with missingness ranging from 0 to 99\%. Some predictors are metadata, and rare tests, so they will not be considered. Ignoring these, the missingness in the useful data ranges from 0 to 65\%. Included in these are 90 different covariates, a few different treatments, as well as survival endpoints (which are all observed). The data can be broken down into a few broad categories, as can be seen in table \ref{table:cats}. There are too many covariates to completely explain here, but I've listed the ones relevant to our models in table \ref{table:importantvars}.

\begin{table}[ !ht]
\centering
\begin{tabular}{|c|c|}
\hline
Type                                                                            & Example                                                                       \\ \hline
Subject data                                                                    & Age range, race, date of birth                                                \\ \hline
Breast Cancer data                                                                     & TNM staging, histology, receptor status                                            \\ \hline
\begin{tabular}[c]{@{}c@{}}Pre brain mets\\ data\end{tabular}                   & Treatment types                                                               \\ \hline
\begin{tabular}[c]{@{}c@{}}Post brain mets\\ clinical observations\end{tabular} & Seizures, headache, nausea                                                    \\ \hline
\begin{tabular}[c]{@{}c@{}}Post brain mets\\ data\end{tabular}                  & Treatment type, type of brain mets						 \\ \hline
Survival data                                                                   & Survival time after brain mets, censoring indicator                           \\ \hline
\end{tabular}
\caption{Data categories and examples}
\label{table:cats}
\end{table}


\begin{table}[!ht]
\centering
\adjustbox{max height=\dimexpr\textheight-5.5cm\relax,
           max width=\textwidth}{

\begin{tabular}{|c|c|c|}
\hline
Name        & \begin{tabular}[c]{@{}c@{}}Percent \\ Missing\end{tabular} & Meaning                                                                                                                                             \\ \hline
capeothno   & 18\%                                                        & \begin{tabular}[c]{@{}c@{}}Indicator: Capecitabine, other, or no chemotherapeutic\\ treatment. Treatment variable 1\end{tabular}                     \\ \hline
lapatrasno  & 18\%                                                        & \begin{tabular}[c]{@{}c@{}}Indicator: Lapatinib, Trastuzumab, or no HER2-directed treatment.\\ Treatment variable 2\end{tabular}                             \\ \hline
controlled  & 12\%                                                        & Indicator: Systemic disease (extracranial) not progressing at the time of diagnosis of brain mets                                                                                                   \\ \hline
%her2        & 10\%                                                        & Indicator: HER2 receptor status                                                                                                                     \\ \hline
hrher2      & 5\%   													& \begin{tabular}[c]{@{}c@{}}Categorical: The hormonal receptor (er,pr) and \\ HER2 receptor status of the subject\end{tabular}  						\\ \hline
braintype   & 4\%                                                        & Categorical: Single, multiple, Leptomeningeal disease                                                                                               \\ \hline                                                                       
timedx      & 1\%                                                        & \begin{tabular}[c]{@{}c@{}}Time (years) from breast cancer diagnosis to brain mets\\ \end{tabular} \\ \hline
site5       & 1\%                                                     & Indicator: First metastasis was to brain                                                                                                            \\ \hline
race2       & 0\%                                                          & Categorical: White, Black, Hispanic, Other                                                                                                          \\ \hline
priorn      & 0\%                                                          & \begin{tabular}[c]{@{}c@{}}Number of prior treatments in metastatic setting \\ before brain mets\end{tabular}                            \\ \hline
os          & 0\%                                                          & Overall survival (months)                                                                                                                           \\ \hline
dead        & 0\%                                                          & Indicator: Censoring indicator                                                                                                                          \\ \hline
agebrainmet & 0\%                                                        & Age at time of brain mets                                                                                        \\ \hline

\end{tabular}
}
\caption{Table of important covariates to be used in the analysis}
\label{table:importantvars}
\end{table}

%%%666

To get a feel for how the data is missing, in figure \ref{fig:missingplot} we see a plot of the missingness in the data. The data is organized in the same fashion as the categories in table \ref{table:cats}. There are some covariates such as genetic tests and clinical observations that are nearly never collected (the nearly solid white lines and blocks), but overall, there seems to be no real pattern in the way the data is missing.
\begin{figure}[h!]
  \centering
    \includegraphics[width=0.8\textwidth]{missingvalues_plot.png}
  \caption{Visualization of missingness in the cancer dataset}
\label{fig:missingplot}
\medskip
\small
Along the horizontal axis are the subjects, and the vertical axis are the covariates. Black denotes observed values whereas white is missing. The covariates with the highest missingness are genetic measures, as well as some clinical assessments.
\end{figure}

This data is exemplary for demonstrating thesis ideas because it is a large retrospective study (pulled from a database), survival amenable, has missingness that is a prime candidate for imputation, and has treatment variables that are not given in an RCT.

Our first step is to clearly define what questions we would like answered. We will consider two separate questions in this analysis. In the first, we will explore the treatment effect (in terms of survival) of Capecitabine (a chemotherapeutic agent) versus other chemotherapeutic agents versus no treatment. In the second, we will look at the effect of two HER2-directed drugs (Lapatinib versus Trastuzumab) versus no HER2-directed drugs in a subset of the patients who are HER2 positive (HER2+).

It isn't vital to understand the entirety of cancer and its treatments for understanding this analysis, but the interested reader may want to look at appendix \ref{app:apdxb} for a very basic overview of breast cancer and how different cancer drugs work. For a much more detailed analysis and other clinically relevant questions, see the upcoming paper by Bugano, Hess, and Berliner. This is the project that this thesis was forked off of.

\section{Imputation}

Our first step in analyzing the data is to impute the missing data. This is a challenging task, because of the attention and care that needs to be given to imputing about 90 covariates with missing data. Although we will not be using all 90 covariates, it is important that we impute them, and do so properly. We need all of the values imputed because: they have the potential to be useful as predictors for other covariates, they might be something we are actually analyzing (now or later), we have spent the money to collect the data, and it strengthens the MAR assumption. As well, it is my opinion (and probably a consensus among applied statisticians) that is better to have too many covariates than not enough. After all, variable selection can be performed if there are too many covariates. 

Our data is quite high dimensional, and there are a many binary and categorical variables as well as a handful of strictly non-negative covariates, thus JM imputation seems inappropriate. Instead, FCS models seem better suited. We will be using the R package mice \cite{VanBuuren2011} because it is easy to use yet powerful.  There are other software implementations in different languages (such as PROC MI in SAS, ICE in Stata, and package mi in R), but I found mice to be the most flexible while also being powerful and easy to use.


The first task we need to do is to assess the missing data mechanism. As we have discussed before, there is no formal statistical test to determine what the mechanism is. It is very unlikely that the data is MCAR (which we typically associate with random/accidental deletion), so it is between MAR and MNAR. We have so many different covariates, and it could reasonably be assumed that the missing data we have could be explained by the type of disease, its stage, the subject's age, their standardized assessment, and their survival time, among other things which we have collected. So it would be reasonable to assume that the missing data mechanism is MAR, and thus imputation can be confidently used.
%spineplots here?

Now that the missing data mechanism has been assessed, the imputations need to be set up. For each covariate with missingness, we need to decide the form of the imputation model that will be used for imputation, and what predictors will be used in it. Most of the covariates that needed imputation were either binary or categorical, so the most popular model chosen were logistic regression, multinomial logit regression, or predictive mean matching. The method that was most appropriate for each situation was chosen. The continuous variables were often selected via regression or predictive mean matching. As for the predictors in each imputation model, nearly every reasonable predictor for each missing covariate is used. This was done to bolster the MAR claim, and avoid variable selection. Van Buuren proposes measures called influx and outflux to determine how worthy and connected each covariate will be as a predictor \cite{VanBuuren2012}. This information was used to remove 11 covariates that were very poor for prediction.
\begin{comment}
\begin{table}[ !ht]
\centering
\begin{tabular}{rrrr}
  \hline
 & Percent Observed & influx & outflux \\ 
  \hline
ki67 & 0.17 & 0.83 & 0.12 \\ 
  brca & 0.05 & 0.95 & 0.04 \\ 
  p53 & 0.01 & 0.99 & 0.01 \\ 
  symptoms & 0.59 & 0.37 & 0.35 \\ 
  seizures & 0.58 & 0.38 & 0.34 \\ 
  headaches & 0.58 & 0.38 & 0.34 \\ 
  nausea & 0.58 & 0.38 & 0.34 \\ 
  chocking & 0.58 & 0.38 & 0.34 \\ 
  balance & 0.58 & 0.38 & 0.34 \\ 
  weakness & 0.58 & 0.38 & 0.34 \\ 
  language & 0.58 & 0.38 & 0.34 \\ 
  cognition & 0.57 & 0.38 & 0.34 \\ 
  ECOG & 0.35 & 0.62 & 0.16 \\ 
  gpaecog & 0.35 & 0.62 & 0.16 \\ 
  gpasum & 0.31 & 0.65 & 0.11 \\ 
   \hline
\end{tabular}
\caption{Percent observed and influx and outflux of the worst predictors. Ki67, brca, and p53 are cancer markers, ECOG and GPA are performance scores}
\label{table:flux}
\end{table}
\end{comment}

Once the model and predictor choices were made, the imputations were run and tuned, and checked by trial and error. This took a considerable amount of time, because after every change made, the algorithm needed to be rerun and the convergence and imputations needed to be assessed. As well, changes in model specification were rarely localized to that variable, as they often affected others. 

It took about three weeks to set up and check the models. This was because the number of covariates was large, and checking the imputations after a change was time consuming. It would not take this long for a smaller dataset. Creating valid imputations is a skill that lies somewhere between an art and a science, so it takes the theory to know what to do, and trial and error to see if you've done it correctly.


For the final MI dataset, it was decided to impute $m=50$ datasets and 40 iterations for each. Research by White et. al  says that the number of imputed datasets $m$ should be at least 100 times the percentage of incomplete cases (for the analysis at hand) \cite{White2011a}. The data used in our analyses had about 30\% missingness, so imputing 50 datasets was the chosen number. Mice generally converges quickly (within 5 or 10 iterations), but by setting the number of iterations so high (40), we are guaranteeing that we are drawing from the correct distribution

After the final imputation model for each covariate with missingness has been set up, we need to run it and save the results. For 50 datasets, 40 iterations, the algorithm runs in about 4 hours, and for 50 datasets with 100 iterations, it took 11.5 hours on a computer with 4 GB of RAM and 4 cores. While this seems like a long time, this process only needs to be done once and requires no human interaction while running, so it can be run overnight and then never need to be touched again. The imputations were run for 100 iterations to see how the run time scaled, as well as to check how the chains behaved and to see how the analyses differed. The results between 50 and 100 iterations were very similar. As well, there is hardly any confidence to be gained going from 50 to 100, and having such large objects in memory can be harder to work with. This is why 50 MI datasets, with 40 iterations for each was chosen.

We need to check our final imputations for convergence, reliability, and validity. Convergence is assessed by looking at the trace plots of the imputed value by iteration for each covariate we imputed. According to van Buuren ``the different streams should be freely intermingled with each other, without showing any definite trends. Convergence is diagnosed when the variance between different sequences is no larger than the variance with each individual sequence'' \cite{VanBuuren2011}. This may be hard to visualize all at one time because for each variable, there will be 50 times the number of missing data values for that covariate, so many authors suggest checking the plots of the chain mean and standard deviation by iteration instead. For continues variables, the chains are checked to see how the variance between and within changes at each iteration, as well as to ensure the chains show no pattern.

Diagnosing convergence for binary and categorical variables is a bit tricky, because the mean of a categorical variable is nonsensical, but we can get a good idea of convergence by looking at if the variance remains constant. And for binary variables, knowing how the groups are coded and looking at where the means of the chains are can give us an idea of healthy convergence. For example, in the cancer dataset, the variable HER2 is coded as 1 for being HER2 negative and 2 for positive. In the available cases, the split between being HER2 negative and positive is about 60 to 40. Thus, is we were to look at the average of the chains groups, we would expect it to be around 1.4. Seeing something radically different than this (perhaps a mean of 1.9) might suggest the imputation was done wrong, because the group percentages are not near what they should be. We can assess the convergence of the binary variable also by checking how its variance changes by iteration. Large changes from groups by iteration will be noted as large changes in the variance plots, whereas staying constant indicates that the draws are remaining relatively stable.

Since there are 90 plots to check, they cannot all be shown in this paper. However, each of them have been checked. Some interesting ones may be seen in figure \ref{fig:traceplot1} and \ref{fig:traceplot2}. Looking at these plots, convergence certainly seems to be the case.  Other authors suggest using a more formal statistical tests such as $\hat{R}$ scale reduction factor, but since most of our interesting variables are categorical, this does not make much sense. 

%change these plots
\begin{figure}[!htbp]
  \centering
    \includegraphics[width=0.8\textwidth]{traceplots1.png}
  \caption{Selected convergence plots of continuous variables}
\label{fig:traceplot1}
\end{figure}


\begin{figure}[!htbp]
  \centering
    \includegraphics[width=0.8\textwidth]{traceplots2.png}
  \caption{Selected convergence plots of binary variables}
\label{fig:traceplot2}
\end{figure}


Once the convergence is diagnosed, the validity of the imputations needs to be inspected.  Diagnostic plots and tables are viewed to ensure that the imputed data are realistic. Common plots include density plots of each covariate compared to the complete cases, as well as bivariate scatterplots for imputed variables. For categorical variables, tables and other group checks may be used. A few of the plots have been replicated here in figures \ref{fig:contdensplot}, \ref{fig:contbwplots}, \ref{fig:clhisto}. Once again, all of the plots cannot be displayed in this paper, but some important ones are recreated. As we can see, not all of the imputed data looks like the real data, but for the majority of the plots, the imputed data seem reasonable and look like they could have been real data had they not been missing. 
%%%%%%%%%%
\begin{figure}[!htbp]
  \centering
    \includegraphics[width=0.6\textwidth]{cont_densplot.png}
  \caption{Selected density plots of continuous imputed variables}
\label{fig:contdensplot}
\medskip
\small
Blue indicates the original density, whereas the red are the imputed dataset densities. The spike in the second plot at zero is due to smaller values being imputed to zero
\end{figure}


\begin{figure}[!htbp]
  \centering
    \includegraphics[width=0.5\textwidth]{bw_timemet.png}%
    \includegraphics[width=0.5\textwidth]{bw_timedx.png}%
  \caption{Selected box and whisker plots of continuous imputed variables}
\label{fig:contbwplots}
\medskip
\small
Blue indicates the original box and whisker plot, whereas the red are the imputed dataset box and whisker plots
\end{figure}

\begin{figure}[!htbp]
  \centering
    \includegraphics[width=0.7\textwidth]{cape_lapat_ratio.png}
  \caption{Selected histograms of the ratio of imputed variables}
\label{fig:clhisto}
\end{figure}


%%%%%%%%%% cape_lapat_ratio
Before we move on to the analyses, let's have a look at the breakdown of the MI data. Using the stack method, table \ref{table:chartab} is created. In this table, we can see how the MI data compares to the available case data, broken down by if the subject had any systemic therapy for the brain mets. Systemic therapy refers to any therapy where treatments are given that ``spread throughout the body to treat cancer cells wherever they may be'' \cite{ MSKCC2015}. Both chemotherapy and HER2-directed therapies are systemic therapies. Systemic therapy does not include treatments such as resection or targeted radiation. 
\begin{table}[!ht]
\centering
\adjustbox{max height=\dimexpr\textheight-5.5cm\relax,
           max width=\textwidth}{
\begin{tabular}{|r|c|c|c|c|}
\hline
\multicolumn{1}{|l|}{}                            & \begin{tabular}[c]{@{}c@{}}Sys. Therapy\\ AC\end{tabular} & \begin{tabular}[c]{@{}c@{}}Sys. Therapy\\ MI\end{tabular} & \begin{tabular}[c]{@{}c@{}}No Sys. Therapy\\ AC\end{tabular} & \begin{tabular}[c]{@{}c@{}}No Sys. Therapy\\ MI\end{tabular} \\ \hline
\multicolumn{1}{|l|}{Age (mean,sd)}               & 51.4(10.8)                                                & 51.2(10.9)                                                & 52.7(11.9)                                                   & 52.9(11.4)                                                   \\ \hline
\multicolumn{1}{|l|}{Breast Cancer subtype}       &                                                           &                                                           &                                                              &                                                              \\ \hline
HR+/HER2-                                         & 27\%                                                      & 31\%                                                      & 28\%                                                         & 33\%                                                         \\ \hline
HR+/HER2+                                         & 19\%                                                      & 18\%                                                      & 12\%                                                         & 13\%                                                         \\ \hline
HR-/HER2+                                         & 22\%                                                      & 20\%                                                      & 15\%                                                         & 12\%                                                         \\ \hline
Triple negative                                   & 32\%                                                      & 32\%                                                      & 45\%                                                         & 42\%                                                         \\ \hline
\multicolumn{1}{|l|}{Prior Therapies for stage 4} & 1(0-3)                                                    & 2(0-4)                                                    & 2(0-4)                                                       & 2(0-4)                                                       \\ \hline
\multicolumn{1}{|l|}{Single Brain Lesion}         & 25\%                                                      & 23\%                                                      & 23\%                                                         & 20\%                                                         \\ \hline
\multicolumn{1}{|l|}{Controlled extra-cranial}    & 40\%                                                      & 40\%                                                      & 35\%                                                         & 36\%                                                         \\ \hline
\multicolumn{1}{|l|}{ECOG 0-1}                    & 84\%                                                      & 70\%                                                      & 53\%                                                         & 40\%                                                         \\ \hline
\multicolumn{1}{|l|}{Local Therapy}               &                                                           &                                                           &                                                              &                                                              \\ \hline
Resection Alone                                   & 5\%                                                       & 5\%                                                       & 9\%                                                          & 7\%                                                          \\ \hline
SBRT Alone                                        & 13\%                                                      & 12\%                                                      & 9\%                                                          & 8\%                                                          \\ \hline
WBRT                                              & 60\%                                                      & 59\%                                                      & 52\%                                                         & 53\%                                                         \\ \hline
Resection/SBRT+WBRT                               & 12\%                                                      & 14\%                                                      & 10\%                                                         & 8\%                                                          \\ \hline
No local                                          & 10\%                                                      & 10\%                                                      & 20\%                                                         & 23\%                                                         \\ \hline
\end{tabular}

}
\caption{Characteristics of AC data versus MI data}
\label{table:chartab}
\end{table}



\section{Survival Analysis}

Now that the $m=50$ datasets are imputed, we are ready to run our models on them.  Before we begin though, we should check all of the models we will run on the available cases to make sure that the model assumptions are met, and to get an idea of the importance of each part of the model. The available case models (Kaplan-Meier, Cox) were run before doing multiple imputation, and all of the model assumptions were met. You can see the results of the AC analyses, alongside the MI values throughout this section.


Before we begin the analyses though, we should discuss the analyses at hand. The endpoint we are looking at is overall survival (i.e. freedom from death of any cause). It should be noted that in all of our survival analyses, we will be doing a landmark analysis. Landmark analysis means that we don't start the analysis at time 0, rather, we start it at a time later than 0. Since the brain mets treatment data was determined after the diagnosis of the brain met, it is not appropriate to use them as baseline covariates in the analyses. Doing a landmark analysis at a time when the vast majority of patients would have had their brain mets treatment choices started allows us to escape this problem. After speaking with cancer experts (Dr. Bugano and Dr. Ibrahim), this landmark time was determined to be 2 months. By doing a landmark analysis at 2 months, we are able to use the treatments as baseline covariates.


The first models that we will check in the MI setting are the Kaplan-Meier curves. Non-informative censoring seems to be a valid assumption, as knowing the censoring status seems to provide no information about the censoring time. The pooled KM estimate was found using Rubin's rules, but under a complimentary log-log transform to get the survival curves towards normality \cite{Marshall2009}. The median and confidence interval about it was computed as the median of the upper and lower confidence bands.

For the comparison of the chemotherapeutic drugs, the KM curves can be seen in figure \ref{fig:chemo_km}.  The available case and MI analyses look similar, but the MI data seems to give a lower median survival time for the chemotherapeutic drugs. For not taking any chemotherapeutic drugs though, the median increased.

\begin{comment}
\begin{figure}[!ht]
  \centering
    \includegraphics[width=0.8\textwidth]{chemo_km.png}
  \caption{Landmarked Kaplan-Meier curves for chemotherapeutic drugs, AC and MI data}
\label{fig:chemo_km}
\end{figure}
\end{comment}
 \begin{figure}[h!]
  \centering
\includegraphics[width=.8\textwidth]{cape_km2}
 \caption{Landmarked Kaplan-Meier curves for chemotherapeutic drugs, AC and MI data}
\label{fig:chemo_km}
\end{figure}


For the HER2-directed drugs, the KM curve can be seen in figure \ref{fig:her2_km}. The available case analysis shows that Lapatinib and Trastuzumab are quite close to each other in terms of survival, while having no HER2-directed treatment is much lower. Once again, the MI analyses give lower values for the median survival time.

\begin{figure}[!ht]
  \centering
    \includegraphics[width=0.8\textwidth]{lapat_km}
  \caption{Landmarked Kaplan-Meier curves for HER2-directed drugs, AC and MI data}
\label{fig:her2_km}
\end{figure}

The Kaplan-Meier curves for both questions seem to show the same thing. As compared to the available case analysis, the estimates of survival are a little pessimistic for the treatments, and a little more optimistic for the no treatment group. This is likely due to the fact that a larger proportion of the missing treatments were assigned to no treatment (which echoes the data), causing their survival to go up. As well, the missingness in treatment for those who took chemotherapy and HER2-directed drugs was much higher at early overall survival times than for later times. 


Now that we have a visual of the curves, we would like to see if there is actually a difference between them. Although visually, we can see that in both cases, no treatment seems to be much worse for survival than treatment, we need to formalize it. To do so, the log rank test needs to be run on them. Recall that we are not able to get the exact log rank test because the quantities needed to calculate them don't exist in the MI setting.  As discussed in the methods section, our best option it to get an approximation to the log rank test by using the Wald test on the Cox regression on treatment membership, and use that value as a proxy for the log rank test since they are asymptotically equivalent.  
\begin{comment}
because in doing so, we would need to compute either the likelihood ratio test or score test, both of which would include calculating the risk set, which is not possible in the MI setting. It has been suggested to pool the chi square statistics via methods presented in Marshall et al 2009, but even they say that this method is poor \cite{Marshall2009}.
\end{comment}

The results for the overall test and for each comparison can be seen in table \ref{table:chemo_lr} and \ref{table:her2_lr}. The AC and MI results are quite similar. For Capecitabine vs other chemotherapeutic drugs, we see a relatively high p-value. Depending on the acceptable rate of significance, different conclusions might be drawn between the two analyses.  In this study, the level is $\alpha=.05$, so changing conclusions is not an issue, however, since we are doing multiple comparisons, it may be prudent to use a Bonferroni correction for multiple testing. For the HER2-directed drugs, there certainly seems to be a difference between the treatments and not having any, but there is no evidence that either of the treatments is better than the other.


\begin{table}[!htbp]
\centering
\begin{tabular}{|l|c|c|}
\hline
                & \multicolumn{2}{c|}{Chemo}                         \\ \hline
                & \multicolumn{1}{l|}{AC} & \multicolumn{1}{l|}{MI} \\ \hline
cape/other/none & \textless.0001          & \textless.0001          \\ \hline
cape/other      & 0.0321                  & 0.033                   \\ \hline
cape/none       & 0.00039                 & .0016                   \\ \hline
other/none      & \textless.0001          & \textless.0001          \\ \hline
\end{tabular}
\caption{Chemotherapeutic drugs log rank tests}
\label{table:chemo_lr}
\end{table}

\begin{table}[!htbp]
\centering
\begin{tabular}{|l|c|c|}
\hline
                   & \multicolumn{2}{c|}{HER2}                         \\ \hline
                   & \multicolumn{1}{l|}{AC} & \multicolumn{1}{l|}{MI} \\ \hline
Lapat/Traztuz/none & \textless.0001          & \textless.0001          \\ \hline
Lapat/Trastuz      & .87                     & .81                     \\ \hline
Lapta/none         & .00017                  & .00018                  \\ \hline
Trastuz/none       & \textless.0001          & \textless.0001          \\ \hline
\end{tabular}
\caption{HER2-directed drugs log rank tests}
\label{table:her2_lr}
\end{table}

Now that we have estimate of the survival curve, we may set up a model to observe how changes in some baseline covariates and the treatments change the hazard. We will do this with the Cox Proportional Hazards model. Once we have a baseline model fit and the assumptions met, we can add our treatment variable to see how this affects the hazard. Speaking with the clinicians, they determined that the covariates listed in table \ref{table:importantvars} were clinically relevant for the baseline model. 

We need to make sure that the proportional hazards assumption is met in the available case model to determine if the use of the Cox model is justified. To check, we visually inspect the spline fit to the Schoenfeld residuals over time (figure \ref{fig:ac_schoenfeld}). In the available cases, the assumption of proportional hazards over time seems reasonable, as a straight line could reasonably be fit between the 95\% confidence bands. A chi-square test of correlation also confirms this. 

\begin{figure}[!ht]
  \centering
    \includegraphics[width=0.8\textwidth]{ac_schoenfeld.png}
  \caption{AC Schoenfeld residuals over time}
\label{fig:ac_schoenfeld}
\end{figure}
Now, to check if we have proportional hazards in the MI data, there are two options. The first is to fit a Cox model on the stacked data, but this will not give us accurate confidence bands. A better option is to look at the Schoenfeld residuals plot for each MI dataset individually. However this may be tedius. We can get the same result if we collect the spline fits from each MI dataset, and then superimpose them onto one plot, and assess the shape. This is what is done for the MI data, as can be seen in figure \ref{fig:mi_schoenfeld}. The shapes are very similar to the available cases, so it is reasonable to assume that the proportional hazards assumption holds on the MI data.
\begin{figure}[!h]
  \centering
    \includegraphics[width=0.8\textwidth]{mi_schoenfeld.png}
  \caption{MI Schoenfeld residuals over time}
\label{fig:mi_schoenfeld}
\end{figure}


Now that the proportional hazards assumption is met, we can run the Cox model to obtain the hazard ratios for the factors we want to control for at baseline. A table of the AC and MI analyses can be seen in table \ref{fig:acmicox}.




\begin{table}[!htbp]
\centering
\adjustbox{max height=\dimexpr\textheight-5.5cm\relax,
           max width=\textwidth}{
\begin{tabular}{|c|c|c|c|c|c|c|c|c|}
\hline
                                                       &                                  &      & \begin{tabular}[c]{@{}c@{}}AC \\ n= 845\end{tabular} &                 &  &      &  \begin{tabular}[c]{@{}c@{}}MI \\ n= 1021\end{tabular}                                                  &                                                            \\ \hline
Variable                                               & Contrast                         & HR   & \begin{tabular}[c]{@{}c@{}}95\% \\ CI\end{tabular}   & pvalue          &  & HR   & \begin{tabular}[c]{@{}c@{}}95\% \\ CI\end{tabular} & \begin{tabular}[c]{@{}c@{}}pvalue\\  (t test)\end{tabular} \\ \hline
HR/HER2                                                & -/+ vs. -/-                      & 0.57 & (0.46,0.71)                                          & \textless0.0001 &  & 0.59 & (0.48,0.72)                                        & \textless0.0001                                            \\ \hline
                                                       & +/- vs. -/-                      & 0.66 & (0.54,0.81)                                          & \textless0.0001 &  & 0.63 & (0.52,0.76)                                        & \textless0.0001                                            \\ \hline
                                                       & +/+ vs. -/-                      & 0.4  & (0.31,0.50)                                          & \textless0.0001 &  & 0.4  & (0.32,0.50)                                        & \textless0.0001                                            \\ \hline
Age                                                    & \textgreater 60 vs. \textless 60 & 1.37 & (1.13,1.65)                                          & 0.0011          &  & 1.45 & (1.22,1.72)                                        & \textless0.0001                                            \\ \hline
Dx to BM                                               & \textgreater 6 vs. \textless 6   & 0.66 & (0.54,0.82)                                          & 0.00013         &  & 0.71 & (0.59,0.86)                                        & 0.0002                                                     \\ \hline
First DM                                               & Brain vs. Oth                    & 0.8  & (0.66,0.97)                                          & 0.026           &  & 0.83 & (0.70,0.99)                                        & 0.02                                                       \\ \hline
Race                                                   & Hisp. Vs. White                  & 0.85 & (0.68,1.07)                                          & 0.17            &  & 0.88 & (0.71,1.08)                                        & 0.11                                                       \\ \hline
                                                       & Black vs. White                  & 1.31 & (1.06,1.63)                                          & 0.014           &  & 1.25 & (1.02,1.52)                                        & 0.015                                                      \\ \hline
                                                       & Other vs. White                  & 0.65 & (0.40,1.04)                                          & 0.075           &  & 0.7  & (0.45,1.07)                                        & 0.05                                                       \\ \hline
\begin{tabular}[c]{@{}c@{}}\# prior \\ Rx\end{tabular} & \textgreater2 vs. 0-2            & 1.58 & (1.31,1.91)                                          & \textless0.0001 &  & 1.53 & (1.29,1.82)                                        & \textless0.0001                                            \\ \hline
BM type                                                & Mult. Vs. Single                 & 1.45 & (1.20,1.76)                                          & \textless0.0001 &  & 1.48 & (1.24,1.76)                                        & \textless0.0001                                            \\ \hline
                                                       & LMD vs. Single                   & 1.6  & (1.21,2.13)                                          & 0.001           &  & 1.58 & (1.25,2.00)                                        & \textless0.0001                                            \\ \hline
Sys. Cont.                                             & Yes vs. No                       & 0.71 & (0.61,0.83)                                          & \textless0.0001 &  & 0.73 & (0.63,0.85)                                        & \textless0.0001                                            \\ \hline
\end{tabular}
}
\caption{AC and MI baseline Cox model}
\label{fig:acmicox}
\end{table}

We may then add in our treatment variable to see how they affect the hazard ratio, and see how they change other factors. Adding chemotherapy can be seen in figure \ref{fig:acmichemo}, and for the HER2 positive patients, the HER2-directed drugs can be seen in figure \ref{fig:acmiher2}.

The analysis for the chemotherapeutic in the MI setting is easy, since it applies to all subjects, but the model for the HER2-directed drugs is more difficult, since it is only for patients who are HER2 positive. Both HER2 status and the treatment were imputed. Because of this, special care needed to be taken when the data was subsetted in each MI dataset. There was some variation between the sample size in  each analysis (ranging from 391 to 415), but this was to be expected.

As it has been posited before, the use of chemotherapeutic and HER2-directed drugs seem to greatly reduce the hazard ratio of death in comparison with not taking any treatment. The effect of other covariates in the presence of these drugs seems to change though.
\begin{table}[!ht]
\centering
\adjustbox{max height=\dimexpr\textheight-5.5cm\relax,
           max width=\textwidth}{
\begin{tabular}{|c|c|c|c|c|c|c|c|c|}
\hline
            &                                  &      & \begin{tabular}[c]{@{}c@{}}AC\\ n=745\end{tabular} &                 &  &      &  \begin{tabular}[c]{@{}c@{}}MI \\ n= 1021\end{tabular}                                                  &                                                             \\ \hline
Variable    & Contrast                         & HR   & \begin{tabular}[c]{@{}c@{}}95\% \\ CI\end{tabular} & p-value         &  & HR   & \begin{tabular}[c]{@{}c@{}}95\% \\ CI\end{tabular} & \begin{tabular}[c]{@{}c@{}}p-value \\ (t test)\end{tabular} \\ \hline
HR/HER2     & -/+ vs. -/-                      & 0.62 & (0.49,0.79)                                        & \textless .0001 &  & 0.63 & (0.51,0.77)                                        & \textless .0001                                             \\ \hline
            & +/- vs. -/-                      & 0.65 & (0.53,0.81)                                        & 0.00011         &  & 0.64 & (0.53,0.78)                                        & \textless .0001                                             \\ \hline
            & +/+ vs. -/-                      & 0.41 & (0.31,0.53)                                        & \textless .0001 &  & 0.42 & (0.34,0.53)                                        & \textless .0001                                             \\ \hline
Age         & \textgreater 60 vs. \textless 60 & 1.34 & (1.10,1.64)                                        & 0.0041          &  & 1.44 & (1.21,1.72)                                        & \textless .0001                                             \\ \hline
Dx to BM    & \textgreater 6 vs. \textless 6   & 0.72 & (0.58,0.90)                                        & 0.0032          &  & 0.71 & (0.58,0.86)                                        & 0.00039                                                     \\ \hline
First DM    & Brain vs. Oth                    & 0.77 & (0.63,0.95)                                        & 0.014           &  & 0.81 & (0.68,0.96)                                        & 0.016                                                       \\ \hline
Race        & Hisp. Vs. White                  & 0.77 & (0.61,0.98)                                        & 0.034           &  & 0.86 & (0.69,1.06)                                        & 0.15                                                        \\ \hline
            & Black vs. White                  & 1.29 & (1.02,1.63)                                        & 0.032           &  & 1.23 & (1.01,1.51)                                        & 0.043                                                       \\ \hline
            & Other vs. White                  & 0.76 & (0.47,1.25)                                        & 0.28            &  & 0.7  & (0.45,1.08)                                        & 0.11                                                        \\ \hline
\# prior Rx & \textgreater2 vs. 0-2            & 1.61 & (1.32,1.98)                                        & \textless .0001 &  & 1.53 & (1.28,1.82)                                        & \textless .0001                                             \\ \hline
BM type     & Mult. Vs. Single                 & 1.46 & (1.20,1.78)                                        & 0.00017         &  & 1.51 & (1.27,1.81)                                        & \textless .0001                                             \\ \hline
            & LMD vs. Single                   & 1.45 & (1.04,2.03)                                        & 0.029           &  & 1.41 & (1.11,1.80)                                        & 0.0049                                                      \\ \hline
Sys. Cont.  & Yes vs. No                       & 0.57 & (0.48,0.68)                                        & \textless .0001 &  & 0.69 & (0.59,0.80)                                        & \textless .0001                                             \\ \hline
Chemo       & Cape. vs. none            & 0.69 & (0.53,0.89)                                        & 0.0046          &  & 0.75 & (0.60,0.95)                                        & 0.018                                                       \\ \hline
            & other vs. none                   & 0.52 & (0.42,0.65)                                        & \textless .0001 &  & 0.58 & (0.47,0.71)                                        & \textless .0001                                             \\ \hline
\end{tabular}
}
\caption{AC and MI Cox model with chemotherapeutic treatments}
\label{fig:acmichemo}
\end{table}

\begin{table}[!ht]
\centering
\adjustbox{max height=\dimexpr\textheight-5.5cm\relax,
           max width=\textwidth}{
\begin{tabular}{|c|c|c|c|c|c|c|c|c|}
\hline
             &                                                             &      & \begin{tabular}[c]{@{}c@{}}AC\\ n=292\end{tabular} &                &  &      & \begin{tabular}[c]{@{}c@{}}MI\\ n=between 391 and 415\end{tabular}           &                                                             \\ \hline
Variable     & Contrast                                                    & HR   & 95\% CI                                            & p-value        &  & HR   & 95\% CI     & \begin{tabular}[c]{@{}c@{}}p-value\\  (t test)\end{tabular} \\ \hline
HR/HER2      & +/+ vs. -/+                                                 & 0.65 & (0.49,0.87)                                        & 0.0036         &  & 0.66 & (0.51,0.85) & 0.0015                                                      \\ \hline
Age          & \textgreater 60 vs. \textless 60                            & 1.38 & (0.95,2.01)                                        & 0.092          &  & 1.58 & (1.15,2.18) & 0.0054                                                      \\ \hline
Dx to BM     & \textgreater 6 vs. \textless 6                              & 0.64 & (0.43,0.97)                                        & 0.033          &  & 0.69 & (0.49,0.99) & 0.041                                                       \\ \hline
First DM     & Brain vs. Oth                                               & 0.84 & (0.58,1.20)                                        & 0.34           &  & 0.86 & (0.62,1.17) & 0.34                                                        \\ \hline
Race         & Hisp. Vs. White                                             & 0.69 & (0.46,1.02)                                        & 0.064          &  & 0.76 & (0.53,1.09) & 0.14                                                        \\ \hline
             & Black vs. White                                             & 1.41 & (0.94,2.11)                                        & 0.1            &  & 1.43 & (1.00,2.04) & 0.047                                                       \\ \hline
             & Other vs. White                                             & 0.7  & (0.32,1.53)                                        & 0.38           &  & 0.83 & (0.46,1.52) & 0.55                                                        \\ \hline
\# prior Rx  & \textgreater2 vs. 0-2                                       & 1.88 & (1.34,2.63)                                        & 0.00028        &  & 1.71 & (1.28,2.28) & 0.00028                                                     \\ \hline
BM type      & Mult. Vs. Single                                            & 1.3  & (0.92,1.86)                                        & 0.14           &  & 1.25 & (0.91,1.70) & 0.16                                                        \\ \hline
             & LMD vs. Single                                              & 2.15 & (1.20,3.88)                                        & 0.011          &  & 1.77 & (1.10,2.83) & 0.018                                                       \\ \hline
Sys. Cont.   & Yes vs. No                                                  & 0.73 & (0.55,0.97)                                        & 0.029          &  & 0.78 & (0.60,1.01) & 0.063                                                       \\ \hline
HER2 therapy & \begin{tabular}[c]{@{}c@{}}Lapat vs. \\ none\end{tabular}   & 0.47 & (0.32,0.69)                                        & 0.00015        &  & 0.52 & (0.37,0.75) & 0.00036                                                     \\ \hline
             & \begin{tabular}[c]{@{}c@{}}Trastuz vs. \\ none\end{tabular} & 0.45 & (0.33,0.61)                                        & \textless.0001 &  & 0.51 & (0.38,0.68) & \textless.0001                                              \\ \hline
\end{tabular}
}
\caption{AC and MI Cox model with HER2-directed treatment, on HER2+ subjects}
\label{fig:acmiher2}
\end{table}
In both analyses, going from available case to MI interpretation, we see that the MI analyses are a little more pessimistic about the hazard ratio (i.e. the hazard ratios move closer to 1). However, the trend in reduction of hazard still holds. Other chemotherapeutics seem to reduce the hazard more than Capecitabine, and in HER2+ patients, Trastuzumab seems to decrease the hazard ever so slightly more than Lapatinib does.


\section{Causal Analysis}

Lastly, we will want to draw causal inference, and see what the average treatment effect of each drug is.  We will do so by assessing the reduction in hazard via the Cox model. In the previous section, we fit a Cox model, but were unable to draw causal conclusions from it since the data was not obtained via an RCT. In this section, we aim to remove the effects of confounding so that we may treat it like an RCT.  Recall that since we are in the MI setting, and that the treatment variables themselves are imputed, we will be using the ``within'' method for propensity scores, as described by Mitra and Reiter \cite{Mitra2012}. For each MI dataset, we will weight each observation by its inverse probability of treatment weight (IPTW), and then run a Cox regression to observe the treatment effects. 

Before we even begin, we should ensure that the assumptions for Rubin's causal model are met. For SUTVA, it is unlikely that a subject's potential outcome will be affected by what treatment another subject receives. As for the no unmeasured confounders assumption, we aim to satisfy this by including many known confounders in the propensity score model.

In order to observe the results from a causal lens, we need to determine what type of treatment effect to look for. It should be noted that some authors use the term marginal to describe population level average effect, and conditional to describe the individual level average effect \cite{Austin2013}. The goal of this study is to make inference about the population, not the individual, so the individual treatment effect is inappropriate. There are two options to choose from in what type of inference to do. There is the ATE, and the ATT. Since we have three comparisons for each comparison, and there is no clear cut accepted treatment for each, the ATT is not as useful as the ATE, thus we will chose to observe the ATE.  Since the ATE will be obtained in each MI dataset by IPTW weighted Cox regression, we are able to pool the results via Rubin's rules to get the ATE as well as the uncertainty associated with it. In the end, we will get an estimate of the true causal relative change in hazard.

The idea for this part of the analysis is to use propensity score weighting to create a balanced sample for each MI dataset where the baseline covariates are not confounded with treatment assignment or result, and then treat it as if it was an RCT. Traditional methods to obtain the propensity score (like logistic regression) cannot be optimized for balance, however machine learning algorithms can be. The method chosen to fit the propensity scores are called Generalized Boosting Models (GBM).  In this ensemble method, ``�the model consists of many simple regression trees iteratively combined to create an overall piecewise constant function. The iterative fitting algorithm begins with a single simple regression tree, and at each new iteration, another tree is added. The new tree is chosen to provide the best fit to the residuals of the model from the previous iteration'' \cite{McCaffrey2013}. After each iteration in the GBM algorithm, the IPTW's are applied to the data, and the balance of the covariates (according to the KS statistic) is measured. Standardized bias is more of a measure of central tendency of the groups, whereas the KS statistic is more about the distribution as the whole. Because of this, we will want to optimize on the KS statistic, although both measures should appear balanced after weighting. Once the GBM algorithm is run, the iteration that minimizes the KS statistic and maximizes its p-value for each variable while not overfitting the data is selected. We then take the propensity scores at that iteration as the propensity scores for that specific MI dataset.

In traditional propensity score analysis, there is only one treatment and control. However in our situation, there are actually three groups in each analysis. So we can no longer use methods tailored to two classes. We could fit a multinomial model (that gives probability for three treatment classifications), but the theory behind this is not as strong and easy to understand as with the binary case.  However, if we make treatment status an indicator (for example 1 if treatment is Capecitabine, and 0 if it is other or none), then we may treat it as if it were binary. Because we have three groups though, we will need to check the balance between all groups (i.e groups Capecitabine vs other, Capecitabine vs none, other vs none) to ensure balance is achieved. Since we are summing three probabilities that are not from the same model, we cannot be guaranteed that the probabilities will sum to unity. This is not an issue though, because our primary interest lies in making sure the groups are balanced, not in the sums of the probabilities \cite{McCaffrey2013}. 

We now have the propensity score model selected, and a plan about how to check the balance, but now we need to ask what pretreatment covariates should we control for. In speaking with Dr. Hess, we determined that there are many pretreatment differences in the population that should be controlled for. So even though we wanted to keep the propensity score model parsimonious, we were not able to. The covariates we deemed to be important to balance before treatment include: breast cancer stage at first diagnosis, race, type of breast cancer, breast cancer surgery type, first metastatic site, number of prior treatments, ECOG score, localized brain mets treatment, type of brain met, age at brain mets, and control of brain mets at diagnosis.

Once the propensity scores have been computed for each MI dataset, we need to check that they are in fact valid. McCaffrey et Al. describe two conditions that need to be checked, and graphical checks to do so. The first condition is to ensure that all of the propensity scores are between 0 and 1. If any propensity score is 0 or 1, then there is no chance that the subject looks like that of another group. If this is the case, then the results will be poor, because the counterfactual framework and propensity score analysis are not meant to handle cases like this \cite{McCaffrey2013}.  All 50 plots for the distribution of the propensity scores cannot be shown in this paper, but I show the propensity scores from two of the MI datasets. In these plots (figure \ref{fig:pshisto}), we see that the propensity scores are never exactly 0 or 1, and as well, the groups are differentiated, yet still have common support.
\begin{figure}[!htbp]
\centering
\includegraphics[width=.5\textwidth]{ps_histo1}%
\includegraphics[width=.5\textwidth]{ps_histo2}
  \caption{Selected propensity score distribution checks}
\label{fig:pshisto}
\end{figure}

The next check is the check of no unmeasured confounders. This is impossible to assert, but should be reasonable if we include any and all factors we believe to be confounding. It can be checked through plots and tables that assess balance. To check for balance, we can check how the KS statistic and standardized bias change after weighting. Two plots are selected, as can be seen in figure \ref{fig:psbal}. This plot shows that after weighting, the standardized bias decreases for nearly every variable to an acceptable level (below about .2 or .25 is considered balanced in the literature). As well, it shows that the KS p-values are increasing, indicating that there is no difference between the distributions of the pretreatment covariates after weighting.


\begin{figure}[!htbp]
\centering
\includegraphics[width=.8\textwidth]{ps_bal_diag}
  \caption{Selected propensity score balance checks}
\label{fig:psbal}
\medskip
\small
This figure shows how the balance is achieved for two different MI datasets. The left hand side denotes the maximum standardized bias for each of the two MI datasets for the pretreatment covariates in all groups. The right hand side shows the minimal p-value for the KS test among all groups in each of the two MI datasets.
\end{figure}


Once propensity score validity is checked, the weights are put into the Cox model for each MI dataset. We cannot be sure that our propensity scores are the truth (this is a known drawback of propensity score analysis), but we can be more confident that it is right by also including the pretreatment covariates as adjustments to the Cox model (along with weighting). This is known as being doubly robust estimator \cite{Lunceford2004}, or adjusting for the confounders.

The results for the unadjusted propensity score weighted and doubly robust adjusted methods can be seen in tables \ref{tab:chemoonly}, \ref{chemorobust}, \ref{tab:lapatonly}, and \ref{lapatfull}. As it can be seen, when only accounting for the treatment effect alone, the estimate is quite unstable, but as we add in the other pretreatment covariates (many of which were actually in the original Cox model in the survival section), the results become more stable. The results are similar in trend to those obtained in the survival section. However, the interpretation is different. These results should be read as if the treatments were administered in a setting where the groups were similar at baseline (i.e. like an RCT). When removing all confounding factors, there seems to be a significant difference between Capecitabine and other chemotherapeutic drugs, but there seems to be no difference between Lapatinib and Trastuzumab in HER2+ patients. All drugs are, however, better than not taking any at all.

%%%%
 
\begin{table}[!htbp]
 \scalebox{0.75}{
\centering
\begin{tabular}{|c|c|c|c|c|c|c|c|c|c|c|c|}
\hline
               & \multicolumn{2}{c|}{AC Unweighted} &  & \multicolumn{2}{c|}{AC IPTW} &  & \multicolumn{2}{c|}{MI unweighted} &  & \multicolumn{2}{c|}{MI IPTW} \\ \hline
               & HR           & 95\% CI             &  & HR        & 95\% CI          &  & HR           & 95\% CI             &  & HR         & 95\% CI         \\ \hline
Cape. vs none  & 0.396        & (0.325,0.482)       &  & 0.655     & (0.481,0.894)    &  & 0.484        & (0.400,0.585)       &  & 0.702          & (0.543,0.906)           \\ \hline
Other vs none  & 0.336        & (0.287,0.394)       &  & 0.567     & (0.46,0.754)     &  & 0.413        & (0.354,0.481        &  & 0.593          & (0.470,0.748)           \\ \hline
Cape. vs other & 1.179        & (0.983,1.416)       &  & 1.156     & (0.966,1.383)    &  & 1.173        & (0.981,1.402)       &  & 1.183          & (0.998,1.404)           \\ \hline
\end{tabular}}
\caption{Unadjusted Chemotherapeutic ATE with IPTW, AC and MI}
\label{tab:chemoonly}
\end{table}

\begin{table}[!htbp]
 \scalebox{0.75}{
\centering
\begin{tabular}{|c|c|c|c|c|c|c|c|c|c|c|c|}
\hline
               & \multicolumn{2}{c|}{AC Unweighted} &  & \multicolumn{2}{c|}{AC IPTW} &  & \multicolumn{2}{c|}{MI unweighted} &  & \multicolumn{2}{c|}{MI IPTW} \\ \hline
               & HR           & 95\% CI             &  & HR        & 95\% CI          &  & HR           & 95\% CI             &  & HR        & 95\% CI          \\ \hline
Cape. vs none  & 0.687        & (0.530,0.891)       &  & 0.603     & (0.443,0.820)    &  & 0.752        & (0.595,0.952)       &  & 0.701     & (0.536,0.917)    \\ \hline
Other vs none  & 0.521        & (0.416,0.653)       &  & 0.452     & (0.340,0.602)    &  & 0.579        & (0.474,0.707)       &  & 0.532     & (0.416,0.681)    \\ \hline
Cape. vs other & 1.318        & (1.078,1.612)       &  & 1.334     & (1.109,1.604)    &  & 1.300        & (1.076,1.570)       &  & 1.317     & (1.100,1.579)    \\ \hline
\end{tabular}}
\caption{Chemotherapeutic ATE, doubly robust adjusted IPTW estimate, AC and MI}
\label{chemorobust}
\end{table}

\begin{table}[!htbp]
 \scalebox{0.75}{
\centering
\begin{tabular}{|l|c|c|c|c|c|c|c|c|c|c|c|}
\hline
                   & \multicolumn{2}{c|}{AC Unweighted}                              & \multicolumn{1}{l|}{} & \multicolumn{2}{c|}{AC IPTW}                                    & \multicolumn{1}{l|}{} & \multicolumn{2}{c|}{MI Unweighted}                              & \multicolumn{1}{l|}{} & \multicolumn{2}{c|}{MI IPTW}                                    \\ \hline
                   & HR                         & 95\% CI                            &                       & HR                         & 95\% CI                            &                       & HR                         & 95\% CI                           &                       & HR                         & 95\% CI                            \\ \hline
Lapat. vs none     & 0.467                      & (0.355,0.616)                      &                       & 0.571                      & (0.381,0.855)                      &                       & 0.474                      & (0.362,0.622)                     &                       & 0.485                      & (0.304,0.775)                      \\ \hline
Trastuz. vs none   & 0.488                      & (0.398,0.597)                      &                       & 0.566                      & (0.421,0.759)                      &                       & 0.506                      & (0.417,0.614)                     &                       & 0.480                      & (0.313,0.735)                      \\ \hline
Lapat. vs Trastuz. & \multicolumn{1}{l|}{0.958} & \multicolumn{1}{l|}{(0.693,1.324)} & \multicolumn{1}{l|}{} & \multicolumn{1}{l|}{1.009} & \multicolumn{1}{l|}{(0.680,1.496)} & \multicolumn{1}{l|}{} & \multicolumn{1}{l|}{0.927} & \multicolumn{1}{l|}{(0.673,1.28)} & \multicolumn{1}{l|}{} & \multicolumn{1}{l|}{1.011} & \multicolumn{1}{l|}{(0.763,1.338)} \\ \hline
\end{tabular}}
\caption{Unadjusted HER2-directed ATE with IPTW, AC and MI}
\label{tab:lapatonly}
\end{table}

\begin{center}
\begin{table}[!htbp]
\scalebox{0.75}{
\begin{tabular}{|c|c|c|c|c|c|c|c|c|c|c|c|}
\hline
                   & \multicolumn{2}{c|}{AC Unweighted} &  & \multicolumn{2}{c|}{AC IPTW} &  & \multicolumn{2}{c|}{MI Unweighted} &  & \multicolumn{2}{c|}{MI IPTW} \\ \hline
                   & HR          & 95\% CI              &  & HR        & 95\% CI          &  & HR           & 95\% CI             &  & HR         & 95\% CI            \\ \hline
Lapat. vs none     & 0.468       & (0.316,0.692)        &  & 0.514     & (0.331,0.798)    &  & 0.524        & (0.367,0.747)       &  & 0.410      & (0.257,0.652)      \\ \hline
Trastuz. vs none   & 0.447       & (0.328,0.6089)       &  & 0.456     & (0.328,0.632)    &  & 0.511        & (0.381,0.685)       &  & 0.388      & (0.249,0.602)      \\ \hline
Lapat. vs Trastuz. & 1.048       & (0.704,1.560)        &  & 1.128     & (0.726,1.754)    &  & 1.026        & (0.713,1.477)       &  & 1.057      & (0.788,1.417)      \\ \hline
\end{tabular}}
\caption{HER2-directed ATE, doubly robust adjusted IPTW estimate, AC and MI}
\label{lapatfull}
\end{table}
\end{center}

%%%
\begin{comment}

This is what will happen for one dataset, but since we are in the MI setting, we have 50. The within method discussed before will be our combination plan. This method is chosen since our treatment variable is itself imputed, so this method makes more sense. The idea is to calculate the weighted cox model for each dataset, and then pool. In order to do this though, we need to check the <<balance achieved on each dataset>>. However, once the balance is assessed, we may just pool the results via rubin's rules. 

After verifying and running the propensity score analysis, the <<results may be seen here>>. As we can see, adjusting for the covariates <<does something>>

!!!EVERYTHING BELOW THIS IS OLD AND WILL PROBABLY GO!!
treatment status, with the Q variables to get our propensity score. We can look at the <standardized bias> before and after the weighting to ensure that we have controlled properly, and to see if we may go forward. Assuming that we have removed the confounding factors and now have two groups that we can treat like it was an RCT, we may now run our Cox model again, but weight by the IPTW. Once we have done this, we can <observe the results from the AC analysis>.

Now we need to apply this propensity score weighting to the MI data. We discussed before the within and across method, and remarked that we were confined to use the within method since our treatment variable (lapat/cape) was itself imputed. So the plan will be to fit the Cox models with the inverse propensity score weights discussed in the AC analysis. We need to be sure that the IPTW weighting is still valid in the MI setting though, so we check <standard biased, other things>. Now we may then pool the results via Rubin's rules and analyze is through the Rubin causal model framework. <the results can be seen here>. The results that we can draw from this are X,Y,Z
\end{comment}
\begin{comment}
#mdplot
image(is.na(data), main = "Missing Values", xlab = "Subject",
      ylab = "Variable", 
      xaxt = "n", yaxt = "n", bty = "n",col=c("green","red"))
axis(1, seq(0, 1, length.out = nrow(data)), 1:nrow(data), col = "white")
axis(2,tick=FALSE,labels=FALSE)

#traceplots
plot(imp50x40,c("timemet","timedx","ki67"),main="Traceplots, continuous")
plot(imp50x40,c("controlled","anysys","HER2"),main="Traceplots, binary")
#fluxtable
xtable(subset(j,influx>.3&outflux<.5,select=c("pobs","influx","outflux")))

#chemo km
par(mfcol=c(2,1))
#plot it
#available case
plot(survfit(Surv(os,dead)~capeothno,subset=(os>2),data=data),
     mark.time = FALSE,col=c("red","blue","green"),
     main="Available case OS for chemo\n 2 month landmark",
     xlab="time (months)",ylab="Surv Prob",
     xlim=c(0,150),ylim=c(0,1),lwd=2,xaxt='n')
axis(1, at=c(0,20,40,60,80,100,120,140,160))

legend("topright", 
       legend = c("Cape.,med=12.3(10.3,15.1)",
                  "other,med=14.5(12.4,16.1)",
                  "none,med=5.4(4.5,6.6)"),
       col=c("red","blue","green")
       ,lwd=2,cex=.7)

#mi

plot(capeothno_1_mtx[,1],pooled_KM_est1,type="s",col="red",
     main="MI OS for chemo \n 2 month landmark",
     xlab="time (months)",ylab="Surv Prob",
     xlim=c(0,150),ylim=c(0,1),lwd=2,xaxt='n')
axis(1, at=c(0,20,40,60,80,100,120,140,160))
lines(capeothno_1_mtx[,1],pooled_KM_est2, col="blue",type="s",lwd=2)
lines(capeothno_1_mtx[,1],pooled_KM_est3, col="green",type="s",lwd=2)



legend("topright", legend = c(paste0("Cape.,med=",round(median_time1,1),"(",round(lower_time1,1),",",round(upper_time1,1),")"),
                              paste0("Other, med=,",round(median_time2,1),"(",round(lower_time2,1),",",round(upper_time2,1),")"),
                              paste0("None,med=",round(median_time3,1),"(",round(lower_time3,1),",",round(upper_time3,1),")")),
       col=c("red","blue","green"),lwd=2,cex=.7)

#km for her2
par(mfcol=c(2,1))
#plot it

#available case
plot(survfit(Surv(os,dead)~lapatrasno,subset=(os>2&HER2==1),data=data),
     mark.time=FALSE,
     main="AC - OS for HER2 therapy \n 2 month landmark",
     xlab="time (months)",ylab="Surv Prob",
     xlim=c(0,150),ylim=c(0,1),xaxt='n',lwd=2,col=c("red","blue","green"))
axis(1, at=c(0,20,40,60,80,100,120,140,160))


legend("topright", 
       legend = c("Lapat., med=20.7(15.4,38.8)",
                  "Trastuz., med=20.3(17.8,26.4)",
                  "none, med=10.0(8.7,13.6)"),
       col=c("red","blue","green")
       ,lwd=2,cex=.7)


#MI
plot(lapatrasno_1_mtx[,1],pooled_KM_est1,type="s",col="red",
     main="MI - OS for HER2 therapy \n 2 month landmark",
     xlab="time (months)",ylab="Surv Prob",
     xlim=c(0,150),ylim=c(0,1),xaxt='n',lwd=2)
axis(1, at=c(0,20,40,60,80,100,120,140,160))

lines(lapatrasno_1_mtx[,1],pooled_KM_est2, col="blue",type="s",lwd=2)
lines(lapatrasno_1_mtx[,1],pooled_KM_est3, col="green",type="s",lwd=2)



legend("topright", legend = c(paste0("Lapat., med=",round(median_time1,1),"(",round(lower_time1,1),",",round(upper_time1,1),")"),
                              paste0("Trastuz., med=",round(median_time2,1),"(",round(lower_time2,1),",",round(upper_time2,1),")"),
                              paste0("None, med=",round(median_time3,1),"(",round(lower_time3,1),",",round(upper_time3,1),")")),
       col=c("red","blue","green"),lwd=2,cex=.7)

\end{comment}
