\section{Cancer and Treatment Overview}
\label{app:apdxb}
%\section{HER2 and Cancer Drugs}

Cancer is a disease in mammals where cells in the body begin to grow in an uncontrolled manner \cite{Cooper1992}. There are many different types of cancers for the many different types of tissues we have in our bodies. This paper focuses on breast cancer.

Breast cancer is a common type of cancer with many different subclassifications, that affects both men and women, although women much more \cite{Cooper1992}. It can be inherited, but often can be detected early on with screening and self-examination. Most cases of breast cancer are sporadic, but studies have shown that a woman�s risk to get breast cancer is doubled if a first degree relative has breast cancer (i.e. it is genetic). When breast cancer is inherited, it often presents itself earlier in life \cite{Morris1998}.  One of the major risks of breast cancer is that it will metastasize, that is, the cancerous cells move from the breast to another area of the body. It should be noted that when a cancer metastasizes to another part of the body, the patient is said to have the original cancer metastasized to a new area, not a new cancer. For example, in this paper we study breast cancer patients who have metastases to the brain, not brain cancer patients. The reason for this is because the makeup of the cancer cell is the same type of tissue as the original location, not the new one.

Luckily, there are lots of different types of treatments for breast cancer and its metastases. I will list the major types here. 
\begin{comment}
(http://www.cancer.gov/types/breast/patient/breast-treatment-pdq)
\end{comment}
\begin{itemize}
\item Chemotherapy is a class of drugs given to cancer patients. These types of drugs target fast growing cells (like cancer) and kill them. Capecitabine is a common chemotherapeutic drug
\item HER2-directed therapy: HER2 is a human protein that is associated with cell growth. If it is determined that the patient has the HER2 protein, then HER2-directed therapies can be used. In HER2-directed therapies, a drug is given that targets the HER2 protein and tries to stop its effect . Common HER2 directed therapies include Lapatinib and Trastuzumab, which we discuss in this paper.
\item Radiation therapy: When the cancer tumor is radiated by precisely located beams in hopes of killing or disturbing the cancer growth process by destroying the cancer DNA.
\item Surgery/resection: A doctor goes in and physically removes the cancerous cells.
\item Hormone therapy: Some cancers have hormone receptors in the tumor cells. The two main receptors are Estrogen Receptors (ER) and Progesterone Receptors (PR). These receptors pick up hormonal signals that tell the cancer cell to grow. If the patient is ER+ or PR+, then drugs that interfere with these receptors can be used
\end{itemize}
Often times, a combination of these treatments is used. The exact course of treatment is very dependent on the type of cancer and the type of person. For example, chemotherapy is very difficult on the body, so it is not often used on the very elderly and frail. The course of treatment given should be determined by a subject matter expert (the oncologist), and is highly individual and cancer dependent.
There have been many books and articles written about these, and for more information, you can check out \cite{Cooper1992} and \cite{Morris1998}.
