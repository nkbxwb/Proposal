\chapter{Methods}
%\section{Motivation}
%\label{sec:Motivation}
I want the framework we use to be easy to use and understand, so that it can easily be discussed among clinicians and other non statistically minded people. On the same token, I want the methods and theory to be sound from a statistical point of view. For the three parts that we are combining, I try to pick the optimal software packages and theory to achieve this. 
Our first decision comes as to what paradigm we should impute under. There are two main divisions in multiple imputation, and they are joint modelling and full conditional specification. In joint modelling, we assume that the data can be described by a multivariate distribution, and draw imputations from a joint distribution of the unknowns for the rows, given what we do know and their associated hyperparameters. There has been extensive research on using the normal model for this, and research shows that it even performs well under non-normality.
On the other hand, there is fully conditional specification. In this paradigm, missing data is imputes on a variable by variable case on the columns, based off of a specification of the imputation model for each imputed variable. 
We are going to have to specify something, there is no escaping that, but I think that it is easier for the average person (especially a clinician) to be able to define a single distribution rather than to guess at a multivariate. In addition, in survival, we will naturally have time variables be only positive, whereas others can take any value. Trying to fit a parametric distribution with these stipulations will be very had, so we will be relegated to using a general distribution (like the normal), which may not fit very well.
Now that we have choose the paradigm, we need to select an implementation of it. Many exist (such as MICE, mi, etc.ect). I wanted to select the implementation that combined ease of use, understanding, and programming. What I decided upon was a method called MICE- Multiple imputation via chained equations. (SOURCE). MICE is a MCMC method that under compatibility, is a gibbs sampler, where we obtain samples from the joint by sampling from the full conditionals. Compatibility means that knowing the full conditionals allows you to know the joint. Specifically <<See 4.5.4 in van burren>> The user defines the full conditionals, so it is possible that the joint may only exist implicitly, and not actually have a functional form. This is an obvious disadvantage of this method, but multiple studies have shown that MICE is robust to this flaw, even when using highly incompatible models. Mice has been shown in simulation studies to yield unbiased estimates with proper coverage
Mice works as follows ��.!!!!!!....
